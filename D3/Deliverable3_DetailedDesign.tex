\documentclass[]{article}

% Imported Packages
%------------------------------------------------------------------------------
\usepackage{amssymb}
\usepackage{amstext}
\usepackage{amsthm}
\usepackage{amsmath}
\usepackage{enumerate}
\usepackage{fancyhdr}
\usepackage[margin=1in]{geometry}
\usepackage{graphicx}
\usepackage{extarrows}
\usepackage{setspace}
\usepackage{placeins}
\usepackage{tabularx}
%------------------------------------------------------------------------------

% Header and Footer
%------------------------------------------------------------------------------
\pagestyle{plain}  
\renewcommand\headrulewidth{0.4pt}                                      
\renewcommand\footrulewidth{0.4pt}                                    
%------------------------------------------------------------------------------

% Title Details
%------------------------------------------------------------------------------
\title{Deliverable \#3 Template}
\author{SE 3A04: Software Design II -- Large System Design}
\date{}                               
%------------------------------------------------------------------------------

% Document
%------------------------------------------------------------------------------
\begin{document}

\maketitle	

\section{Introduction}
\label{sec:introduction}
% Begin Section

This section should provide an brief overview of the entire document.

\subsection{Purpose}
\label{sub:purpose}
% Begin SubSection
\begin{enumerate}[a)]
	\item Delineate the purpose of the document
	\item Specify the intended audience for the document
\end{enumerate}
% End SubSection

\subsection{System Description}
\label{sub:system_description}
% Begin SubSection
\begin{enumerate}[a)]
	\item Give a brief description of the system. This could be a paragraph or two to give some context to this document.
\end{enumerate}
% End SubSection

\subsection{Overview}
\label{sub:overview}
% Begin SubSection
\begin{enumerate}[a)]
	\item Describe what the rest of the document contains 
	\item Explain how the document is organised
\end{enumerate}

% End SubSection

% End Section

\section{State Charts for Controller Classes}
\label{sec:state_charts_for_controller_classes}
% Begin Section
\begin{figure}
	\centering
	\includegraphics[scale=0.6]{UIController-StatechartDiagram.png}
	\caption{Statechart Diagram for UIController}
	\label{fig:uicontroller_statechart}
\end{figure}
\begin{figure}
	\centering
	\includegraphics[scale=0.6]{TempoController-StatechartDiagram.png}
	\caption{Statechart Diagram for TempoController}
	\label{fig:uicontroller_statechart}
\end{figure}
\begin{figure}
	\centering
	\includegraphics[scale=0.6]{LyricController-StatechartDiagram.png}
	\caption{Statechart Diagram for LyricController}
	\label{fig:uicontroller_statechart}
\end{figure}
\begin{figure}
	\centering
	\includegraphics[scale=0.6]{ArtistController-StatechartDiagram.png}
	\caption{Statechart Diagram for ArtistController}
	\label{fig:uicontroller_statechart}
\end{figure}
\FloatBarrier
% End Section

\section{Sequence Diagrams}
\label{sec:sequence_diagrams}
% Begin Section
\begin{figure}
	\centering
	\includegraphics[scale=0.8]{TempoInput-SequenceDiagram.png}
	\caption{Sequence Diagram for the Tempo Input}
	\label{fig:tempo_input_sequence}
\end{figure}

\begin{figure}
	\centering
	\includegraphics[scale=0.8]{LyricInput-SequenceDiagram.png}
	\caption{Sequence Diagram for the Lyric Input}
	\label{fig:lyric_input_sequence}
\end{figure}

\begin{figure}
	\centering
	\includegraphics[scale=0.8]{ArtistInput-SequenceDiagram.png}
	\caption{Sequence Diagram for the Artist Input}
	\label{fig:artist_input_sequence}
\end{figure}
\FloatBarrier
% End Section

\section{Detailed Class Diagram}
\label{sec:detailed_class_diagram}
% Begin Section
This section should provide a detailed class diagram for your application.
% End Section

\appendix
\newpage
\section{Division of Labour}
\label{sec:division_of_labour}
% Begin Section
\noindent\begin{tabular}{l l}
	\textbf{Jemar Jones} & \\
	\textbf{Rendavid Dimen} & \\
	\textbf{Samraj Nalwa} & \\
	\textbf{Samuel Scargall} & \\
	\textbf{Spencer Park} & Created the statechart diagram in figure \ref{fig:uicontroller_statechart} and sequence diagrams in figures \ref{fig:tempo_input_sequence}, \ref{fig:lyric_input_sequence}, and \ref{fig:artist_input_sequence}.\\
	\textbf{Stephan Arulthasan} & \\
\end{tabular}
\\

\noindent Agreement with the Division of Labour outline above:

\noindent\begin{tabularx}{\linewidth}{|l|X|}
	\hline
	\rule{0pt}{2em} \textbf{Jemar Jones} & \\
	\hline
	\rule{0pt}{2em} \textbf{Rendavid Dimen} & \\
	\hline
	\rule{0pt}{2em} \textbf{Samraj Nalwa} & \\
	\hline
	\rule{0pt}{2em} \textbf{Samuel Scargall} & \\
	\hline
	\rule{0pt}{2em} \textbf{Spencer Park} & \\
	\hline
	\rule{0pt}{2em} \textbf{Stephan Arulthasan} & \\
	\hline
\end{tabularx}

\newpage
\section*{IMPORTANT NOTES}
\begin{itemize}
	\item You do \underline{NOT} need to provide a text explanation of each diagram; the diagram should speak for itself
	\item Please document any non-standard notations that you may have used
	\begin{itemize}
		\item \emph{Rule of Thumb}: if you feel there is any doubt surrounding the meaning of your notations, document them
	\end{itemize}
	\item Some diagrams may be difficult to fit into one page
	\begin{itemize}
		\item It is OK if the text is small but please ensure that it is readable when printed
		\item If you need to break a diagram onto multiple pages, please adopt a system of doing so and throughly explain how it can be reconnected from one page to the next; if you are unsure about this, please ask me
	\end{itemize}
	\item Please submit the latest version of Deliverable 1 and Deliverable 2 with Deliverable 3
	\begin{itemize}
		\item They do not have to be a freshly printed versions; the latest marked versions are OK
	\end{itemize}
	\item If you do \underline{NOT} have a Division of Labour sheet, your deliverable will \underline{NOT} be marked
\end{itemize}


\end{document}
%------------------------------------------------------------------------------